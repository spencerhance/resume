%% start of file `template.tex'.
%% Copyright 2006-2013 Xavier Danaux (xdanaux@gmail.com).
%
% This work may be distributed and/or modified under the
% conditions of the LaTeX Project Public License version 1.3c,
% available at http://www.latex-project.org/lppl/.


\documentclass[11pt,letterpaper,sans]{moderncv}        
% possible options include font size ('10pt', '11pt' and '12pt'), paper size 
%('a4paper', 'letterpaper', 'a5paper', 'legalpaper', 'executivepaper' and 
%'landscape') and font family ('sans' and 'roman')

% modern themes
% style options are 'casual' %(default), 'classic','oldstyle' and 'banking'
\moderncvstyle{banking}




% color options 'blue' 
%(default), 'orange', 'green', 'red', 'purple', 'grey' and 'black'
\moderncvcolor{blue}                                
%\renewcommand{\familydefault}{\sfdefault}         
% to set the default font; use '\sfdefault' for the default sans serif font, 
%'\rmdefault' for the default roman one, or any tex font name

% Suppress page numbers
\nopagenumbers{}

% character encoding
\usepackage[utf8]{inputenc}                       % if you are not using xelatex ou lualatex, replace by the encoding you are using
%\usepackage{CJKutf8}                              % if you need to use CJK to typeset your resume in Chinese, Japanese or Korean

% adjust the page margins
\usepackage[scale=0.90]{geometry}
%\setlength{\hintscolumnwidth}{3cm}                % if you want to change the width of the column with the dates
%\setlength{\makecvtitlenamewidth}{10cm}           % for the 'classic' style, if you want to force the width allocated to your name and avoid line breaks. be careful though, the length is normally calculated to avoid any overlap with your personal info; use this at your own typographical risks...

\usepackage{import}

%
% Remove the space after the address
%
\makeatletter
\patchcmd{\makehead}{%search
  \flushmakeheaddetails\@firstmakeheaddetailselementtrue\\\null}{%replace
  \flushmakeheaddetails\@firstmakeheaddetailselementtrue\par\vspace{-\baselineskip}\null}{%success
  }{%failure
  }
\makeatother
%
%
%


% personal data
\name{Spencer}{Hance}
%\title{Computer Engineering}                               
%\address{234 Commonwealth Ave}{Boston -- MA -- 02116}%{ 
%Permanent Address: 156 Minuteman Road -- Ridgefield -- CT -- 06877}
\phone[mobile]{(203) 240-8072}                  
% \phone[fixed]{01234 123456}                    
%\phone[fax]{+3~(456)~789~012}                      
\email{shance@ece.neu.edu}    
%\extrainfo{\emailsymbol\emaillink{spencerhance@gmail.com}}
% optional, remove / comment the line if not wanted
\homepage{www.shance.me}                         
\extrainfo{Availability: May - August 2018}
% optional, remove / comment the line if not wanted
%\photo[64pt][0.4pt]{picture}                       % optional, remove / comment the line if not wanted; '64pt' is the height the picture must be resized to, 0.4pt is the thickness of the frame around it (put it to 0pt for no frame) and 'picture' is the name of the picture file
%\quote{Some quote}  % optional


%----------------------------------------------------------------------------------
%            content
%----------------------------------------------------------------------------------
\begin{document}

%-----       resume       ---------------------------------------------------------
\makecvtitle

% Bring the education section closer to the title
\vspace{-51pt}

%\small{Bio goes here}




% Education section
\section{Education}
\vspace{1pt}

%\subsection{Academic Qualifications}
%\vspace{5pt}

% Begin list of schools
\begin{itemize}

\item[] {\cventry{December 2018}{Bachelor of Science}{Northeastern 
University}{Boston, MA}{\textit{\textbf{Computer Engineering}}}{}}
\vspace{-1mm}	\begin{itemize}
	 \item IEEE (\textbf{Treasurer Fall'16, Fall'17})
	 \item Beta Gamma Epsilon Engineering Fraternity (\textbf{Vice President Fall'17})
	\end{itemize}
	
\item[] {\textit{Relevant Coursework}}
\vspace{1pt} \begin{itemize}
               \item High Performance Computing (\textbf{C, PThreads, OpenMP, OpenMPI, CUDA}), 
               Software Security, 
               Networks,
               Computer Systems (\textbf{C, x86}),
               Algorithms (\textbf{C++}),
               Digital Logic Design 
               (\textbf{Verilog, FPGA, MIPS}), 
               Embedded Design (\textbf{C, FPGA})
              \end{itemize}

%\vspace{2pt}
%\item{\cventry{June 2014}{AP Scholar with Distinction}{Ridgefield High
%School}{Ridgefield, CT}{}{}}
%\vspace{-1mm}	\begin{itemize}
%	 \item Boy Scouts of America: Life Scout 
%	 \item Badminton Club: President (2012-2014)
%	 \item Western CT Youth Orchestra: Principal Second-Violin
%	\end{itemize}

\end{itemize}
%\vspace{2pt}
\vspace{-6pt}




% Experience Section
\section{Work Experience}

\vspace{1pt}

% Begin main list
\begin{itemize}

% AMD round 2
\item[] {\cventry{January 2018 -- Present}{Research Co-op - CPU Architecture}
{Advanced Micro Devices (AMD) (Research)}{Boxborough, MA}{}{\vspace{1pt}}
\vspace{-8pt}	\begin{itemize}
	 \item Researching cache prefetching methods for HPC/Exascale workloads
	 \item Profiled HPC applications with dynamic instrumentation \textbf{(C++, Python)} to discover new cache metrics
	\end{itemize}
}
\vspace{6pt}


% MIT LL
\item[] {\cventry{October 2017 -- Present}{Technical Assistant - Machine Learning}
{MIT Lincoln Laboratory}{Cambridge, MA}{}{\vspace{1pt}}
\vspace{-8pt}	\begin{itemize}
	 \item Implementing and documenting Machine Learning pipelines \textbf{(Python, Anaconda, SLURM)}
% 	 \item Evaluating a meta learning system to automatically select models
	\end{itemize}
}
\vspace{6pt}

% AMD
\item[] {\cventry{January -- July 2017}{Engineer Co-op - GPU Architecture}
{Advanced Micro Devices (AMD)}{Boxborough, MA}{}{\vspace{1pt}}
\vspace{-8pt}	\begin{itemize}
	 \item Researched new GPU compressed cache designs and presented work at internal innovation expo
	 \item Co-developed cache simulator (\textbf{C++}) and decreased simulator runtime by 300\%
	 \item Designed simulation framework (\textbf{Bash, Python}) to run and analyze large-scale experiments on LSF cluster
% 	 \item Implemented Catch unit testing framework and increased code coverage
	\end{itemize}
}
\vspace{6pt}

% EnerNOC
\item[] {\cventry{January -- December 2016}{Performance Engineering Co-op}
{EnerNOC}{Boston, MA}{}{\vspace{1pt}}
\vspace{-8pt}	\begin{itemize}
	 \item Created automated tests to measure web-application performance \textbf{(JMeter, LoadRunner, Jenkins, AWS)}
	 \item Ported a proprietary algorithm at company hackathon to \textbf{Python/OpenCL} and gained a 7x speedup
	 \item Developed a \textbf{MEAN.js/JavaScript} application to generate and load test data from Hadoop cluster
	 \item Developed a custom status page and \textbf{Splunk} dashboards to display critical performance data on the office wall
% 	 \item Implemented status pages to monitor production services
	\end{itemize}
}
% \vspace{6pt}


% End experience list
\end{itemize}
\vspace{-6pt}



% Research Experience Section
\section{Research Experience}

\vspace{1pt}

% Begin main list
\begin{itemize}

% NUCAR
\item[] {\cventry{July 2015 -- Present}{Student Competitor}
{International Supercomputing Competitions (SC'15, SC'16, ISC'17, SC'17)}{Boston, MA}{}{\vspace{1pt}}
\vspace{-8pt}	\begin{itemize}
		\item Designed/built four supercomputers and optimized scientific applications for them
		\item Competition record of the HPCG benchmark at ISC'17 in Frankfurt, Germany
		\item Won the ``MacGyver Award'' for sourcing and building a HPC cluster in 6 hours at SC'16 due to shipping issues
% 		\item Developed visualizations to show program execution
		\end{itemize}
}
\vspace{6pt}


\item[] {\cventry{October 2014 -- Present}{Undergraduate Researcher}
{NU Computer Architecture Research Group}{Boston, MA}{}{\vspace{1pt}}
\vspace{-8pt}	\begin{itemize}
		\item Contributed significantly to \textbf{open-source} Multi2Sim CPU-GPU simulator for \textbf{C} to \textbf{C++} rewrite
		\item Developed unit tests with Google Test and discovered/fixed bug affecting the accuracy of all emulations 
		\item Designed a tool \textbf{(Bash, Python, SQLite3)} to run many parallel fault-injection simulations on a SLURM cluster
		%\item Utilized \textbf{Python} and SQLite3 to analyze simulation results
		\end{itemize}
		
% \end{itemize}		
}
% \vspace{6pt}

% End experience list
\end{itemize}
\vspace{-6pt}


% Skills section
\section{Technical Skills}

% \vspace{2pt}

\begin{itemize}
\item[]  \textbf{Languages:} Python (Pandas), C(++), Bash
\vspace{3pt}
\item[]  \textbf{Technologies:} Linux, HPC Concepts, LaTeX, Git, Perforce, GDB, LSF, Splunk, JMeter
% \vspace{3pt}
% \item[]  \textbf{Certifications:} CompTIA A+
% \vspace{3pt}
% \item[] \textbf{Hobbies:} Powerlifting, Biking, etc, etc 
%\vspace{-2pt}

\end{itemize}


%Extracurricular section
%\section{Extracurricular}
%
%\begin{itemize}
%\vspace{6pt}
% \item[] \href{multi2sim.org}{\textbf{Multi2Sim}} Worked on a large open-source 
%heterogeneous system simulator.  Tasks included porting the Cache Coherency 
%and AMD Southern Islands model from C to C++, writing Unit Tests using the 
%Google Test platform, and debugging code throughout the simulator (GDB).
%\vspace{6pt}
%
%\end{itemize}


% References
%\begin{center}
%	\textit{References available upon request}
%\end{center}


\end{document}
